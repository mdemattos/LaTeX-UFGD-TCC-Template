\documentclass[a4paper, 12pt]{report}
\usepackage[brazil]{babel}
\usepackage[utf8]{inputenc}
% \usepackage[T1]{fontenc}
\usepackage[pdftex]{hyperref}
\usepackage{setspace}
\usepackage[bmargin=2cm, tmargin=3cm, lmargin=3cm, rmargin=2cm]{geometry}
\usepackage{listings}
\lstset{basicstyle=\ttfamily\normalsize}
\usepackage{url}

\newcommand{\HRule}{\rule{\linewidth}{0.5mm}}

\begin{document}

% Definição do cabeçalho do manual
  \begin{center}
    \HRule \\[0.5cm]
    \fontsize{12}{16}
%     \fontfamily{qpl}%\fontseries{b}%
%     \fontshape{sc} %\headlinecolor
    {\Large \bfseries Manual para modelo de escrita de textos de Monografia para o Curso de BSI}
    \HRule
  \end{center}
  \vspace{0.5cm}
  
  {\it {\bfseries Resumo}. Este artigo tem como objetivo descrever o estilo a ser usado na confeção de 
monografias de TCC para o curso de Bacharelado em Sistemas de Informação na UFGD. É apresentada uma breve 
introdução de como os Trabalhos de Conclusão de Curso devem ser elaborados utilizando o modelo LaTeX para a 
confecção do mesmo. }

\section*{1. Informações gerais}

Para o modelo de Trabalho de Conclusão de Curso foi usado o formato de papel A4, com 3.0 cm para margem 
superior e margem esquerda e 2.0 cm para margem inferior e margem 
direita. A fonte principal utilizada é a padrão do LaTeX, Computer Mordern, tamanho 12. Para o uso do 
modelo, são usados os pacotes babel, inputenc, hyperref, fancyhdr, multicol, float, multirow, graphicx, 
geometry, etoolbox, lmodern e setspace, portanto, é recomendado que baixe esses pacotes individualmente 
antes de usar o modelo.

\section*{2. Estrutura do TCC}

Para a criação de um trabalho de TCC são necessárias diversas páginas obrigatórias tais como folha de 
rosto, resumo, sumário, referências bibliográficas, e caso seja a versão final, folha de aprovação e 
de 
agradecimentos (opcional).

Para criar as páginas do TCC neste modelo\cite{MAT14}, basta utilizar os comandos
\lstinline[breaklines=true]!\maketitle! para criar a capa,  
\lstinline[breaklines=true]!\folhaderosto! para criar a folha de rosto, 
\lstinline[breaklines=true]!\folhadeaprovacao! para a folha de aprovação (se necessário) e 
\lstinline[breaklines=true]!\tableofcontents! para o sumário. Para criar uma página de 
agradecimentos, basta abrir o arquivo {\bfseries agradecimentos.tex} e inserir o texto.

Para criar o resumo de seu trabalho, basta abrir o arquivo {\bfseries resumo.tex} e inserir o resumo na 
parte indicada. Similarmente, para inserir a introdução, os capítulos do trabalho e a conclusão, basta 
abrir os arquivos 
repectivos aos capítulos {\bfseries introducao.tex}, {\bfseries cap1.tex}, {\bfseries cap2.tex}, {\bfseries 
conclusao.tex} e inserir o texto no lugar do texto de exemplo.

\section*{3. Imagens e Tabelas}

As imagens e tabelas, junto com suas legendas, devem estar centralizadas, com sua legenda logo abaixo 
fazendo referência a mesma. Exemplos de como criar imagens e tabelas são fornecidades nos capítulos de 
exemplo do modelo.

\section*{4. Referências bibliográficas}

Para criar as referências bibliográficas, é necessário criar um arquivo com a extensão .bib, e inserir o 
tipo de referência no formato BibTeX. Uma maneira simples de obter referências no formato BibTeX é acessar 
o Google Acadêmico\cite{GOOG}, ir na aba de configurações e ativar a opção ``Mostre links para importar 
citações para o BibTeX''. Depois de salvar as configurações, basta pesquisar o trabalho que deseja 
referenciar e escolher a opção ``Importe para o BibTeX''.

Para citar algum trabalho que já esteja referenciado é necessário usar o comando 
\lstinline[breaklines=true]!\cite{}! e colocar entre as chaves o nome de identificação da referência 
bibliográfica.


  \bibliographystyle{acm}
  \bibliography{manual}


\end{document}
