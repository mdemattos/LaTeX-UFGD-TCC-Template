\chapter{A história do Rock'n roll}

\section{A Origem negra}


A origem de um estilo musical difundido por todos os cantos do planeta não 
haveria de ter uma 
explicação fácil, afinal, foi longo o caminho necessário para que o rock pudesse nascer.

Diversos ritmos e comportamentos foram se adaptando com o tempo e, em uma pura combinação de 
fatores, surgiu primeiro o {\it rhythm and blues} e depois o {\it rock and roll} propriamente 
dito. Uma retrospectiva pelas raízes é necessária para que se possa entender sua importância no 
cenário, não apenas musical, mas também social do mundo.

O {\it rhythm and blues} é a vertente negra do Rock. É ali que vamos buscar, quase que exclusivamente 
as origens corpóreas do Rock. Reprimidos pela sociedade wasp ({\it white, anglo-saxon and protestant}), 
a mão-de-obra negra, desde os tempos da escravidão, se refugiava na música (os {\it blues}) e na dança 
para dar vazão, pelo corpo, ao protesto que as vias convencionais não permitiam. \cite{CHA1985}

Em suas origens, o rock and roll era essencialmente uma música afro-americana. Os ritmos 
sincronizados, a voz rouca e sentimental e as vocalizações de chamado-e-resposta características dos 
trabalhadores negros eram parte da herança da música africana e tornaram-se tijolos com os quais o 
{\it rock and roll} foi construído. \cite{FRI1996}.

\section{Um pitada de folk e country}

Embora, grande parte da população branca dos Estados Unidos não aceitasse a música dançante dos 
negros, o {\it rythim e blues} ia conquistando admiradores. Não apenas os negros poderiam extravasar suas 
angústias e 
tristezas se divertindo com o novo e frenético som. Agora, os brancos queriam participar e também 
fizeram sua contribuição para o nascimento do {\it rock and roll}.

As músicas {\it folk} e {\it country} dos brancos criavam baladas sobre o cotidiano de pessoas comuns. 
Assim 
como o {\it rhythm and blues} negro, que até esse momento não se misturava, a música {\it country} branca 
também 
buscava 
manifestar suas experiências e emoções e representava uma alternativa às canções melosas e rimadas 
das músicas populares da época.

Assim como a música transmitia a emoção do artista, o público respondia, na mesma medida, movendo 
seus corpos em vibrações que acompanhavam o movimento dos artistas \cite{FRI1996} O 
{\it rock and roll} era, para muitos, um catalisador de identidade para os adolescentes que, criados por 
pais hierarquicamente influenciados pela estrutura do exército, do trabalho e da família, não 
queriam obedecer a regras apenas porque elas existiam. Queriam seguir o rumo que suas próprias vidas 
os levariam.

Apenas alguns anos após o término da II Guerra Mundial, a juventude americana, ainda traumatizada 
pelas perdas humanas — principalmente de jovens –, queria, após anos de sofrimento, se divertir. 
Músicas despretensiosas, ritmos dançantes e o clima de festa serviriam para alegrar tanto os 
músicos, quanto os ouvintes.

Nesse contexto, ocorreu a mistura de música branca e negra que, mais alguns anos depois, já na 
década de 1960, percebendo que já viviam miscigenados através do {\it rock and roll}, vão reclamar e 
protestar contra o racismo.
Foi dessa forma, por meio da festa e da diversão, que brancos e negros aprenderam a dançar e cantar 
juntos.

\section {O rockabilly pede licença: finalmente a mistura se completa}

O {\it rhythm and blues}, originado do {\it blues} (rural e urbano), da música gospel e do {\it jump band 
jazz}, surgiu para 
os 
negros, popularizou-se e espalhou-se.
O {\it folk} e o {\it country} dos brancos se modernizaram e passaram a ser tocados nas rádios. Aos poucos, 
quase que imperceptivelmente, os dois caminhos começaram a se aproximar e, alguns jovens, ansiosos 
por sair da monótona música
popular americana, decidiram criar uma nova estrutura de som e ritmo.

“Em meados do anos 50, alguns jovens, influenciados por Williams, ansiavam por mais. Cientes da 
força e da emocionalidade do {\it rhythm and blues}, eles quiseram incorporar ‘a batida’ à autêntica 
música country. Elvis Presley – nascido no Mississipi e depois estabelecido em Memphis – entrou na 
gravadora {\it Sun Records} em uma tarde de julho para gravar um {\it blues} rural intitulado {\it That’s 
All 
Right 
(Mama)}.

Gravado com apenas um violão, uma guitarra, um baixo e cantado com trêmulo e displicente abandono, 
Elvis criou a síntese do {\it country/blues/rhythm and blues} conhecida como {\it rockabilly}. Mais tarde, 
a 
bateria 
somou-se 
ao conjunto e o {\it rockabilly} tornou-se um gênero de transição para alguns artistas brancos, atraindo 
astros como Jerry Lee Lewis, Johnny Cash, Carl Perkins e Roy Orbinson para a {\it Sun Records} antes do 
final da década”.
% \subsection{Início dos anos 1950 e 1960}
% 
% \subsubsection{Rock and Roll}
% O rock and roll surgiu nos subúrbios dos Estados Unidos no final dos anos 1940 e início da década de 1950 e rapidamente se espalhou para o resto do mundo.
% No começo, o novo estilo rock sofreu várias críticas negativas e algumas positivas, mas sempre atrapalhando seus trabalhos. Muitos diziam que o "novo" rock 
% incentivava o satanismo. Suas origens imediatas remontam a uma mistura entre blues e country, mas com influência de vários gêneros musicais com o rhythm and
% blues.3 . Em 1951, na cidade de Cleveland (no Estado do Ohio), o discotecário Alan Freed começou a tocar a mistura de blues, country e rhythm and blues para 
% uma plateia multirracial e a ele é creditado a primeira utilização da expressão "rock and roll" para 
% descrever a música \cite{friedlander1996rock}. % arrumar
% 
% 
% \subsection{Era de Ouro}
% 
% No Reino Unido, o movimento trad jazz levou muitos artistas do blues a visitar o país. Enquanto estava desenvolvendo o Concorde,
% o sucesso "Rock Island Line", de Lonnie Donegan, em 1955, foi a principal influência e ajudou a desenvolver uma nova tendência de 
% grupos musicais de skiffle em todo a Grã-Bretanha, incluindo os Beatles. Foi em solo britânico que se desenvolveu uma grande cena rock and roll, 
% sem as barreiras raciais que mantiveram a "gravações de raça" ou rhythm and blues separados nos 
% Estados Unidos\cite{castro2008web}. % arrumar
% 
% \section{Os melhores álbuns de rock de todos os tempos}
% 
% De acordo com ..., esses são os melhores álbuns da música.
% %\flushleft
% \begin{table}[!htb]
%   \centering
%   \caption{Titulo da tabela aqui.}
% \begin{tabular}{|c|c|c|c|}
% \hline
% \textbf{Posição} & \textbf{Nome do Album} & \textbf {Nome da banda} & \textbf {Ano} \\
% \hline
% 1 & Sgt. Pepper's Lonely Hearts Club Band & The Beatles & 1967 \\
% \hline
% 2 & Dark Side of the Moon & Pink Floyd & 1973 \\
% \hline
% 3 & Thriller & Michael Jackson & 1982 \\
% \hline
% 4 & Led Zeppelin IV & Led Zeppelin & 1982 \\
% \hline
% 5 & The Joshua Tree & U2 & 1987 \\
% \hline
% 6 & Exile on Main St. & The Rolling Stones & 1972 \\
% \hline
% 7 & Tapestry & Carole King & 1970 \\
% \hline
% 8 & Highway 61 Revisited & Bob Dylan & 1965 \\
% \hline
% 9 & Pet Sounds & The Beach Boys & 1966 \\
% \hline
% 10 & Revolver & The Beatles & 1966 \\
% \hline
% 11 & Ten & Pearl Jam & 1991 \\
% \hline
% 12 & Abbey Road & The Beatles & 1969 \\
% \hline
% 13 & Supernatural & Santana & 1999 \\
% \hline
% 14 & Metallica & Metallica & 1991 \\
% \hline
% 15 & Born to Run & Bruce Springsteen & 1975 \\
% \hline
% 16 & Music from the Motion Picture ``Purple Rain'' & Prince & 1984 \\
% \hline
% 17 & Back in Black & AC/DC & 1980 \\
% \hline
% 18 & Let it Bleed & The Rolling Stones & 1969 \\
% \hline
% 19 & The Doors & The Doors & 1967 \\
% \hline
% 20 & America Beauty & Grateful Dead & 1970 \\
% \hline
% 21 & Come on Over & Shania Twain & 1997 \\
% \hline
% 22 & Who's Next & The Who & 1971 \\
% \hline 
% 23 & Songs in the Key of Life & Stevie Wonder & 1976 \\
% \hline
% 24 & Rumours & Fletwood Mac & 1977 \\
% \hline
% 25 & The Wall & Pink Floyd & 1979 \\
% \hline
% 26 & Jagged Little Pill & Alanis Morissette & 1995 \\
% \hline 
% 27 & Come Away with Me & Norah Jones & 2002 \\
% \hline
% 28 & The Marshall Mathers LP & Eminem & 2000 \\
% \hline
% 29 & Speakerboxx/The Love Below & OutKast & 2003 \\
% \hline
% 30 & The Chronic & Dr. Dre & 1992 \\
% \hline
% 31 & Licensed do Ill & Beastie Boys & 1986 \\
% \hline
% 32 & Appetite for Destruction & Guns N' Roses & 1987 \\
% \hline
% 33 & Wide Open Spaces & Dixie Chicks & 1998 \\
% \hline
% 34 & Kind of Blue & Miles Davis & 1959 \\
% \hline
% 35 & Hotel California & Eagles & 1976 \\ 
% \hline
% 36 & Hysteria & Def Leppard & 1987 \\
% \hline 
% 37 & Grease & Vários artistas & 1977 \\
% \hline
% 38 & What's Going On & Marvin Gaye & 1971 \\
% \hline
% 39 & White Album & The Beatles & 1968 \\
% \hline
% 40 & Saturday Night Fever & Vários artistas & 1977 \\
% \hline
% % 41 & Are You Experienced & Jimi Hendrix & 1967 \\
% % \hline
% % 42 & Revolver & The Beatles & 1966 \\
% % \hline
% % 43 & Boston & Boston & 1976 \\
% % \hline 
% % 44 & Slippery When Wet & Bon Jovi & 1986 \\
% % \hline
% % 45 & Achtung Baby & U2 & 1991 \\
% % \hline
% % 46 & Whitney Houston & Whistney Houston & 1985 \\
% % \hline
% % 47 & Led Zeppelin II & Led Zeppelin & 1969 \\
% % \hline
% % 48 & Crash & Dave Matthews Band & 1996 \\
% % \hline
% % 49 & Sticky Fingers & The Rolling Stones & 1971 \\
% % \hline
% % 50 & Dookie & Green Day & 1994 \\
% 
% \end{tabular}
% \end{table}